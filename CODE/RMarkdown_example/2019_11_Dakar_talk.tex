% Options for packages loaded elsewhere
\PassOptionsToPackage{unicode}{hyperref}
\PassOptionsToPackage{hyphens}{url}
%
\documentclass[
  ignorenonframetext,
]{beamer}
\usepackage{pgfpages}
\setbeamertemplate{caption}[numbered]
\setbeamertemplate{caption label separator}{: }
\setbeamercolor{caption name}{fg=normal text.fg}
\beamertemplatenavigationsymbolsempty
% Prevent slide breaks in the middle of a paragraph
\widowpenalties 1 10000
\raggedbottom
\setbeamertemplate{part page}{
  \centering
  \begin{beamercolorbox}[sep=16pt,center]{part title}
    \usebeamerfont{part title}\insertpart\par
  \end{beamercolorbox}
}
\setbeamertemplate{section page}{
  \centering
  \begin{beamercolorbox}[sep=12pt,center]{part title}
    \usebeamerfont{section title}\insertsection\par
  \end{beamercolorbox}
}
\setbeamertemplate{subsection page}{
  \centering
  \begin{beamercolorbox}[sep=8pt,center]{part title}
    \usebeamerfont{subsection title}\insertsubsection\par
  \end{beamercolorbox}
}
\AtBeginPart{
  \frame{\partpage}
}
\AtBeginSection{
  \ifbibliography
  \else
    \frame{\sectionpage}
  \fi
}
\AtBeginSubsection{
  \frame{\subsectionpage}
}
\usepackage{amsmath,amssymb}
\usepackage{lmodern}
\usepackage{iftex}
\ifPDFTeX
  \usepackage[T1]{fontenc}
  \usepackage[utf8]{inputenc}
  \usepackage{textcomp} % provide euro and other symbols
\else % if luatex or xetex
  \usepackage{unicode-math}
  \defaultfontfeatures{Scale=MatchLowercase}
  \defaultfontfeatures[\rmfamily]{Ligatures=TeX,Scale=1}
\fi
\usetheme[]{Shanghai}
% Use upquote if available, for straight quotes in verbatim environments
\IfFileExists{upquote.sty}{\usepackage{upquote}}{}
\IfFileExists{microtype.sty}{% use microtype if available
  \usepackage[]{microtype}
  \UseMicrotypeSet[protrusion]{basicmath} % disable protrusion for tt fonts
}{}
\makeatletter
\@ifundefined{KOMAClassName}{% if non-KOMA class
  \IfFileExists{parskip.sty}{%
    \usepackage{parskip}
  }{% else
    \setlength{\parindent}{0pt}
    \setlength{\parskip}{6pt plus 2pt minus 1pt}}
}{% if KOMA class
  \KOMAoptions{parskip=half}}
\makeatother
\usepackage{xcolor}
\newif\ifbibliography
\usepackage{color}
\usepackage{fancyvrb}
\newcommand{\VerbBar}{|}
\newcommand{\VERB}{\Verb[commandchars=\\\{\}]}
\DefineVerbatimEnvironment{Highlighting}{Verbatim}{commandchars=\\\{\}}
% Add ',fontsize=\small' for more characters per line
\usepackage{framed}
\definecolor{shadecolor}{RGB}{248,248,248}
\newenvironment{Shaded}{\begin{snugshade}}{\end{snugshade}}
\newcommand{\AlertTok}[1]{\textcolor[rgb]{0.94,0.16,0.16}{#1}}
\newcommand{\AnnotationTok}[1]{\textcolor[rgb]{0.56,0.35,0.01}{\textbf{\textit{#1}}}}
\newcommand{\AttributeTok}[1]{\textcolor[rgb]{0.77,0.63,0.00}{#1}}
\newcommand{\BaseNTok}[1]{\textcolor[rgb]{0.00,0.00,0.81}{#1}}
\newcommand{\BuiltInTok}[1]{#1}
\newcommand{\CharTok}[1]{\textcolor[rgb]{0.31,0.60,0.02}{#1}}
\newcommand{\CommentTok}[1]{\textcolor[rgb]{0.56,0.35,0.01}{\textit{#1}}}
\newcommand{\CommentVarTok}[1]{\textcolor[rgb]{0.56,0.35,0.01}{\textbf{\textit{#1}}}}
\newcommand{\ConstantTok}[1]{\textcolor[rgb]{0.00,0.00,0.00}{#1}}
\newcommand{\ControlFlowTok}[1]{\textcolor[rgb]{0.13,0.29,0.53}{\textbf{#1}}}
\newcommand{\DataTypeTok}[1]{\textcolor[rgb]{0.13,0.29,0.53}{#1}}
\newcommand{\DecValTok}[1]{\textcolor[rgb]{0.00,0.00,0.81}{#1}}
\newcommand{\DocumentationTok}[1]{\textcolor[rgb]{0.56,0.35,0.01}{\textbf{\textit{#1}}}}
\newcommand{\ErrorTok}[1]{\textcolor[rgb]{0.64,0.00,0.00}{\textbf{#1}}}
\newcommand{\ExtensionTok}[1]{#1}
\newcommand{\FloatTok}[1]{\textcolor[rgb]{0.00,0.00,0.81}{#1}}
\newcommand{\FunctionTok}[1]{\textcolor[rgb]{0.00,0.00,0.00}{#1}}
\newcommand{\ImportTok}[1]{#1}
\newcommand{\InformationTok}[1]{\textcolor[rgb]{0.56,0.35,0.01}{\textbf{\textit{#1}}}}
\newcommand{\KeywordTok}[1]{\textcolor[rgb]{0.13,0.29,0.53}{\textbf{#1}}}
\newcommand{\NormalTok}[1]{#1}
\newcommand{\OperatorTok}[1]{\textcolor[rgb]{0.81,0.36,0.00}{\textbf{#1}}}
\newcommand{\OtherTok}[1]{\textcolor[rgb]{0.56,0.35,0.01}{#1}}
\newcommand{\PreprocessorTok}[1]{\textcolor[rgb]{0.56,0.35,0.01}{\textit{#1}}}
\newcommand{\RegionMarkerTok}[1]{#1}
\newcommand{\SpecialCharTok}[1]{\textcolor[rgb]{0.00,0.00,0.00}{#1}}
\newcommand{\SpecialStringTok}[1]{\textcolor[rgb]{0.31,0.60,0.02}{#1}}
\newcommand{\StringTok}[1]{\textcolor[rgb]{0.31,0.60,0.02}{#1}}
\newcommand{\VariableTok}[1]{\textcolor[rgb]{0.00,0.00,0.00}{#1}}
\newcommand{\VerbatimStringTok}[1]{\textcolor[rgb]{0.31,0.60,0.02}{#1}}
\newcommand{\WarningTok}[1]{\textcolor[rgb]{0.56,0.35,0.01}{\textbf{\textit{#1}}}}
\setlength{\emergencystretch}{3em} % prevent overfull lines
\providecommand{\tightlist}{%
  \setlength{\itemsep}{0pt}\setlength{\parskip}{0pt}}
\setcounter{secnumdepth}{-\maxdimen} % remove section numbering
\usepackage{amsmath,amsfonts,amssymb}
\usepackage{graphicx}
\usepackage{multimedia}
\usepackage{array}
\usepackage{hyperref}
\usepackage{multimedia}

\def\II{\mathbb{I}}
\def\IN{\mathbb{N}}
\def\IP{\mathbb{P}}
\def\IR{\mathbb{R}}
\def\IZ{\mathbb{Z}}


\def\A{\mathcal{A}}
\def\B{\mathcal{B}}
\def\C{\mathcal{C}}
\def\D{\mathcal{D}}
\def\E{\mathcal{E}}
\def\F{\mathcal{F}}
\def\G{\mathcal{G}}
\def\I{\mathcal{I}}
\def\M{\mathcal{M}}
\def\P{\mathcal{P}}
\def\R{\mathcal{R}}
\def\S{\mathcal{S}}
\def\U{\mathcal{U}}
\def\V{\mathcal{V}}

\def\b0{\mathbf{0}}
\def\bA{\mathbf{A}}
\def\bB{\mathbf{B}}
\def\bC{\mathbf{C}}
\def\bD{\mathbf{D}}
\def\bE{\mathbf{E}}
\def\bF{\mathbf{F}}
\def\bG{\mathbf{G}}
\def\bI{\mathbf{I}}
\def\bL{\mathbf{L}}
\def\bM{\mathbf{M}}
\def\bN{\mathbf{N}}
\def\bP{\mathbf{P}}
\def\bR{\mathbf{R}}
\def\bS{\mathbf{S}}
\def\bU{\mathbf{U}}
\def\bV{\mathbf{V}}


\def\Rzero{\mathcal{R}_0}
\def\Rc{\mathcal{R}_C}
\def\Rt{\mathcal{R}}
\def\human{\textrm{\Gentsroom}}
\def\fit{\mathbb{F}}
\def\birth{\mathcal{B}}
\def\dsum{\mathop{\displaystyle \sum }}%
\def\diag{\mathsf{diag }}
\def\Sp{\textrm{Sp }}

\def\dss{\displaystyle}


\def\nbOne{{\mathchoice {\rm 1\mskip-4mu l} {\rm 1\mskip-4mu l}
{\rm 1\mskip-4.5mu l} {\rm 1\mskip-5mu l}}}

\newtheorem{proposition}[theorem]{Proposition}
\newtheorem{conjecture}[theorem]{Conjecture}

%%%%%%%%%%%%%%%%%
\usepackage{tikz}
\usetikzlibrary{shapes,arrows}
\usetikzlibrary{positioning}
\tikzstyle{cloud} = [draw, ellipse,fill=red!20, node distance=0.87cm,
minimum height=2em]
\tikzstyle{line} = [draw, -latex']
\usetikzlibrary{shapes.symbols,shapes.callouts,patterns}
\usetikzlibrary{calc}


%\usepackage{comicsans}
%\renewcommand{\sfdefault}{comic}
%\usepackage{arevmath}


%%%%
%%%%
%% BEAMER STUFF %%
%%%%
%%%%

%\usetheme{Boadilla}
%\usetheme{CambridgeUS}

%\usetheme{AnnArbor}
%\usecolortheme{crane}

\usetheme{Shanghai}
\usecolortheme{seagull}

%\usetheme{Shanghai}
%\usetheme{rochester}
%\usecolortheme{seagull}
%\useinnertheme[shadow=true]{rounded}
%\useoutertheme{shadow}

%\useoutertheme{smoothbars}
%\usetheme{Trondheim}
%\useoutertheme{shadow}
%\usecolortheme{beetle}

\pgfdeclareimage[height=0.5cm]{logo}{../../FIGS/UM_l_clr_horz_png}
\pgfdeclareimage[height=2.5cm]{titlelogo}{../../FIGS/UM_l_clr_horz_png}
%\logo{\pgfuseimage{titlelogo}}

\setbeamertemplate{theorems}[numbered]

\def\red{\color[rgb]{1,0,0}}
\def\blue{\color[rgb]{0,0,1}}
\def\green{\color[rgb]{0,1,0}}
\setbeamertemplate{navigation symbols}{}
%\setbeamertemplate{navigation symbols}[vertical]


\newenvironment{changemargin}[2]{%
\begin{list}{}{%
\setlength{\topsep}{0pt}%
\setlength{\leftmargin}{#1}%
\setlength{\rightmargin}{#2}%
\setlength{\listparindent}{\parindent}%
\setlength{\itemindent}{\parindent}%
\setlength{\parsep}{\parskip}%
}%
\item[]}{\end{list}}


% Make one image take up the entire slide content area in beamer,.:
% centered/centred full-screen image, with title:
% This uses the whole screen except for the 1cm border around it
% all. 128x96mm
\newcommand{\titledFrameImage}[2]{
\begin{frame}{#1}
%\begin{changemargin}{-1cm}{-1cm}
\begin{center}
\includegraphics[width=108mm,height=\textheight,keepaspectratio]{#2}
\end{center}
%\end{changemargin}
\end{frame}
}

% Make one image take up the entire slide content area in beamer.:
% centered/centred full-screen image, no title:
% This uses the whole screen except for the 1cm border around it
% all. 128x96mm
\newcommand{\plainFrameImage}[1]{
\begin{frame}[plain]
%\begin{changemargin}{-1cm}{-1cm}
\begin{center}
\includegraphics[width=108mm,height=76mm,keepaspectratio]{#1}
\end{center}
%\end{changemargin}
\end{frame}
}

% Make one image take up the entire slide area, including borders, in beamer.:
% centered/centred full-screen image, no title:
% This uses the entire whole screen
\newcommand{\maxFrameImage}[1]{
\begin{frame}[plain]
\begin{changemargin}{-1cm}{-1cm}
\begin{center}
\includegraphics[width=\paperwidth,height=0.99\paperheight,keepaspectratio]
{#1}
\end{center}
\end{changemargin}
\end{frame}
}

% This uses the entire whole screen (to include in frame)
\newcommand{\maxFrameImageNoFrame}[1]{
\begin{changemargin}{-1cm}{-1cm}
\begin{center}
\includegraphics[width=\paperwidth,height=0.99\paperheight,keepaspectratio]
{#1}
\end{center}
\end{changemargin}
}


% Make one image take up the entire slide area, including borders, in beamer.:
% centered/centred full-screen image, no title:
% This uses the entire whole screen
\newcommand{\maxFrameImageColor}[2]{
\begin{frame}[plain]
\setbeamercolor{normal text}{bg=#2!20}
\begin{changemargin}{-1cm}{-1cm}
\begin{center}
\includegraphics[width=\paperwidth,height=\paperheight,keepaspectratio]
{#1}
\end{center}
\end{changemargin}
\end{frame}
}


%% Alternative way for full page
\newcommand{\maxFrameImageNew}[1]{
\begin{tikzpicture}[overlay, remember picture]
\node[minimum width=\paperwidth-\pgflinewidth,minimum height=\paperheight-\pgflinewidth] (engauge) at (current page.north west) [anchor=north west]
    {\includegraphics[width=\paperwidth,keepaspectratio]{#1}};
\end{tikzpicture}
}


\AtBeginSubsection[]
{
\begin{frame}
\tableofcontents[currentsection,currentsubsection]
\end{frame}
}

\AtBeginSection[]
{
\begin{frame}
\tableofcontents[currentsection,currentsubsection]
\end{frame}
}


%% CITY INFORMATION (name, etc.) for later use
\def\IATA{BJS}
\def\IATAs{BJS\ }

\def\FIGURES{..}



\setbeamercolor{normal text}{bg=gray!5}

\makeatletter
\newlength\beamerleftmargin
\setlength\beamerleftmargin{\Gm@lmargin}
\makeatother

\setbeamerfont{title}{size={\fontsize{15}{15}}}
\setbeamerfont{subtitle}{size={\fontsize{15}{15}}}
\setbeamerfont{author}{size={\fontsize{10}{20}}}
\setbeamerfont{institute}{size={\fontsize{8}{10}}}
\setbeamerfont{date}{size={\fontsize{10}{10}}}

\ifLuaTeX
  \usepackage{selnolig}  % disable illegal ligatures
\fi
\IfFileExists{bookmark.sty}{\usepackage{bookmark}}{\usepackage{hyperref}}
\IfFileExists{xurl.sty}{\usepackage{xurl}}{} % add URL line breaks if available
\urlstyle{same} % disable monospaced font for URLs
\hypersetup{
  pdftitle={Epidémiologie mathématique},
  pdfauthor={Julien Arino},
  hidelinks,
  pdfcreator={LaTeX via pandoc}}

\title{Epidémiologie mathématique}
\subtitle{à l'heure du ``Open Data'\,'}
\author{Julien Arino}
\date{26 novembre 2019}
\institute{Department of Mathematics\\
Data Science Nexus\\
Visual and Automatic Disease Analytics training program\\
Centre for Disease Modelling (West)\\
\strut \\
University of Manitoba\\
\strut \\
\texttt{Julien.Arino@umanitoba.ca}}

\begin{document}
\frame{\titlepage}

\begin{frame}[fragile]{Remarques sur ce document}
\protect\hypertarget{remarques-sur-ce-document}{}
Tous les liens devraient être clickables \vfill Le code sera fourni (sur
demande) \vfill Le code est en \texttt{R} .. j'aurais pu utiliser Python
mais je hais ce language :) \vfill Une partie de ces transparents est
dynamique: ils sont produits avec
\href{https://rmarkdown.rstudio.com/index.html}{Rmarkdown} en utilisant
des données extraites du web et à jour au jour de compilation de ce
document (2022-05-02)
\end{frame}

\begin{frame}[fragile]{Pour générer ces transparents}
\protect\hypertarget{pour-guxe9nuxe9rer-ces-transparents}{}
Il vous faudra les programmes (gratuits) suivants

\begin{itemize}
\tightlist
\item
  \texttt{R} (une version réecente \(\geq 3.5\))
  (\href{https://www.r-project.org/}{lien})
\item
  \texttt{RStudio}
  (\href{https://www.rstudio.com/products/rstudio/download/}{lien})
\item
  Une distribution de \LaTeX (\href{https://miktex.org/}{MiKTeX} sous
  Windows, TeX Live sous Linux \& Mac)
\item
  Plusieurs librairies \texttt{R}
\item
  Accès au web (une copie des pages/fichiers est fournie, quand même)
\end{itemize}
\end{frame}

\begin{frame}{Principes directeurs}
\protect\hypertarget{principes-directeurs}{}
Nous vivons dans un monde où les données (data) sont devenues une
ressource très prisée \vfill Beaucoup de données sont accessibles
librement \vfill   Un modélisateur n'est pas \emph{obligé} d'utiliser
des données, mais quand des données sont disponibles, vous devriez au
moins essayer de voir de quoi il retourne \vfill Si vous voulez ``avoir
un impact'\,' (pouvoir influencer une politique de santé publique),
oubliez la stabilité globale!
\end{frame}

\hypertarget{les-donnuxe9es-sont-partout}{%
\section{Les données sont partout}\label{les-donnuxe9es-sont-partout}}

\begin{frame}{Données propriétaires versus données libres (open data)}
\protect\hypertarget{donnuxe9es-propriuxe9taires-versus-donnuxe9es-libres-open-data}{}
\textbf{Données propriétaires}

\begin{itemize}
\tightlist
\item
  souvent générées par des compagnies, gouvernements ou laboratoires de
  recherche
\item
  quand elles sont disponibles, viennent avec multiples restrictions
  \vfill
\end{itemize}

\textbf{Open data}

\begin{itemize}
\tightlist
\item
  souvent générées par les mêmes entités (compagnies, labos) mais
  \emph{libérées} après une certaine période
\item
  de plus en plus fréquent pour les gouvernements/entités publiques
\item
  grande variété de licenses, donc attention
\item
  grande variété de qualités, donc attention
\end{itemize}
\end{frame}

\begin{frame}{Initiatives Open Data}
\protect\hypertarget{initiatives-open-data}{}
Mouvement récent (5-10 ans): governments (locaux ou plus haut) créent
des portails où les données sont centralisées et publiées

\begin{itemize}
\tightlist
\item
  \url{https://data.winnipeg.ca/}
\item
  \url{https://open.alberta.ca/opendata}
\item
  \url{https://open.canada.ca/en/open-data}
\item
  \url{https://data.europa.eu/euodp/data/}
\item
  \url{http://data.un.org/}
\item
  \url{https://data.worldbank.org/}
\item
  \url{https://www.who.int/gho/database/en/}
\end{itemize}
\end{frame}

\begin{frame}[fragile]{Méthodes de récupération des données}
\protect\hypertarget{muxe9thodes-de-ruxe9cupuxe9ration-des-donnuxe9es}{}
\begin{itemize}
\tightlist
\item
  A la main
\item
  En utilisant des programmes comme
  \href{http://markummitchell.github.io/engauge-digitizer/}{Engauge
  Digitizer} ou \href{https://github.com/pn2200/g3data}{g3data}
\item
  En utilisant des API
\item
  En utilisant des processeurs de langage naturel (web scraping)
\item
  En utilisant des packages (\texttt{R}, \texttt{Python} principalement)
\end{itemize}
\end{frame}

\begin{frame}[fragile]{Exemple: la population du Sénégal}
\protect\hypertarget{exemple-la-population-du-suxe9nuxe9gal}{}
\footnotesize

\begin{Shaded}
\begin{Highlighting}[]
\ControlFlowTok{if}\NormalTok{ (}\ConstantTok{FALSE}\NormalTok{) \{}
\NormalTok{    pop\_data\_CTRY }\OtherTok{\textless{}{-}} \FunctionTok{wb}\NormalTok{(}\AttributeTok{country =} \FunctionTok{c}\NormalTok{(}\StringTok{"SEN"}\NormalTok{), }\AttributeTok{indicator =} \StringTok{"SP.POP.TOTL"}\NormalTok{,}
        \AttributeTok{mrv =} \DecValTok{100}\NormalTok{)}
\NormalTok{\} }\ControlFlowTok{else}\NormalTok{ \{}
\NormalTok{    pop\_data\_CTRY }\OtherTok{=} \FunctionTok{readRDS}\NormalTok{(}\StringTok{"pop\_data\_CTRY\_downloaded.Rds"}\NormalTok{)}
\NormalTok{\}}
\NormalTok{y\_range }\OtherTok{=} \FunctionTok{range}\NormalTok{(pop\_data\_CTRY}\SpecialCharTok{$}\NormalTok{value)}
\NormalTok{y\_axis }\OtherTok{\textless{}{-}} \FunctionTok{make\_y\_axis}\NormalTok{(y\_range)}
\FunctionTok{pdf}\NormalTok{(}\AttributeTok{file =} \StringTok{"pop\_SEN.pdf"}\NormalTok{, }\AttributeTok{width =} \DecValTok{11}\NormalTok{, }\AttributeTok{height =} \FloatTok{8.5}\NormalTok{)}
\FunctionTok{plot}\NormalTok{(pop\_data\_CTRY}\SpecialCharTok{$}\NormalTok{date, pop\_data\_CTRY}\SpecialCharTok{$}\NormalTok{value }\SpecialCharTok{*}\NormalTok{ y\_axis}\SpecialCharTok{$}\NormalTok{factor,}
    \AttributeTok{xlab =} \StringTok{"Année"}\NormalTok{, }\AttributeTok{ylab =} \StringTok{"Population"}\NormalTok{, }\AttributeTok{type =} \StringTok{"b"}\NormalTok{, }\AttributeTok{lwd =} \DecValTok{2}\NormalTok{,}
    \AttributeTok{yaxt =} \StringTok{"n"}\NormalTok{)}
\FunctionTok{axis}\NormalTok{(}\DecValTok{2}\NormalTok{, }\AttributeTok{at =}\NormalTok{ y\_axis}\SpecialCharTok{$}\NormalTok{ticks, }\AttributeTok{labels =}\NormalTok{ y\_axis}\SpecialCharTok{$}\NormalTok{labels, }\AttributeTok{las =} \DecValTok{1}\NormalTok{, }\AttributeTok{cex.axis =} \FloatTok{0.8}\NormalTok{)}
\NormalTok{dave }\OtherTok{\textless{}{-}} \FunctionTok{dev.off}\NormalTok{()}
\end{Highlighting}
\end{Shaded}

\normalsize
\end{frame}

\begin{frame}{}
\protect\hypertarget{section}{}
\maxFrameImageNoFrame{pop_SEN.pdf}
\end{frame}

\hypertarget{exemple-graphiose-de-lorme}{%
\section{Exemple -- Graphiose de
l'orme}\label{exemple-graphiose-de-lorme}}

\begin{frame}{Graphiose de l'orme}
\protect\hypertarget{graphiose-de-lorme}{}
\begin{itemize}
\tightlist
\item
  Maladie fongique de l'orme \vfill
\item
  Causée par le champignon \emph{Ophiostoma ulmi} \vfill
\item
  Transmise par le scolyte de l'orme (Scolytus scolytus) (coléoptère)
  \vfill
\item
  A décimé les forêts urbaines en amérique du nord
\end{itemize}
\end{frame}

\begin{frame}{}
\protect\hypertarget{section-1}{}
\maxFrameImageNoFrame{../../FIGS/WinnipegOpenDataPortal}
\end{frame}

\begin{frame}{}
\protect\hypertarget{section-2}{}
\maxFrameImageNoFrame{../../FIGS/WODTreeMap}
\end{frame}

\begin{frame}{}
\protect\hypertarget{section-3}{}
\maxFrameImageNoFrame{../../FIGS/WODTreeMapZoom}
\end{frame}

\begin{frame}[fragile]{Recuperation des données des arbres}
\protect\hypertarget{recuperation-des-donnuxe9es-des-arbres}{}
\footnotesize

\begin{Shaded}
\begin{Highlighting}[]
\NormalTok{allTrees }\OtherTok{=} \FunctionTok{read.csv}\NormalTok{(}\StringTok{"https://data.winnipeg.ca/api/views/hfwk{-}jp4h/rows.csv?accessType=DOWNLOAD"}\NormalTok{)}
\end{Highlighting}
\end{Shaded}

\normalsize

Voilà ce que cela retourne:

\begin{Shaded}
\begin{Highlighting}[]
\FunctionTok{dim}\NormalTok{(allTrees)}
\end{Highlighting}
\end{Shaded}

\begin{verbatim}
## [1] 291601     17
\end{verbatim}
\end{frame}

\begin{frame}[fragile]{On nettoie un peu}
\protect\hypertarget{on-nettoie-un-peu}{}
\footnotesize

\begin{Shaded}
\begin{Highlighting}[]
\NormalTok{elms\_idx }\OtherTok{=} \FunctionTok{grep}\NormalTok{(}\StringTok{"American Elm"}\NormalTok{, allTrees}\SpecialCharTok{$}\NormalTok{Common.Name, }\AttributeTok{ignore.case =} \ConstantTok{TRUE}\NormalTok{)}
\NormalTok{elms }\OtherTok{=}\NormalTok{ allTrees[elms\_idx, ]}
\end{Highlighting}
\end{Shaded}

\normalsize ce qui nous laisse 47273 ormes américains
\end{frame}

\end{document}
